\documentclass[runningheads,a4paper]{article}
\usepackage[utf8]{inputenc}
\setcounter{tocdepth}{3}
\usepackage[english]{babel} 
\usepackage{graphicx}
\usepackage{grffile}
\usepackage{float}
\usepackage{multicol}
\usepackage{url}
\usepackage{titling}
\usepackage[hidelinks]{hyperref}
\setcounter{secnumdepth}{5}
%Margins
\usepackage[
margin=2cm,
includefoot
]{geometry}
\graphicspath{{img/}}
%Headers and Footers
\usepackage{fancyhdr}
\pagestyle{fancy}
\fancyhead{}
\fancyfoot{}
\fancyfoot[R]{\thepage}
\renewcommand{\headrulewidth}{0pt}
\renewcommand{\footrulewidth}{0pt}
\setlength\parindent{24pt}
\begin{document}
%Title Page
\begin{titlepage}
\begin{center}
\includegraphics[width=10cm]{UP.jpg}  \\
[1cm]
\line(1,0){300} \\
[0.3cm]
\textsc{\Large
NavUP \\
Testing Phase \\
\hfill \break 1 May 2017
%University of Pretoria
}\\
[0.1cm]
\line(1,0){300} \\
[0.7cm]
\textsc{\Large
Team GladIOS Navigation
} \\
\end{center}
\begin{center}
\begin{multicols}{2}
\textsc{\large\\
Jacobus Marais\\ 
u15188397\\ 
}
\textsc{\large\\
Mia Gerber\\
u15016502\\ 
}
\textsc{\large\\
Victor Twigge\\
u10376802\\ 
}
\columnbreak
\textsc{\large\\
Nicaedin Suklal\\
u15207812\\
}
\textsc{\large\\
Darren Adams\\
u14256232\\
}
\textsc{\large\\
Nkosinathi Mothoa\\
u12077420\\
}
\end{multicols}
\textsc{	\\ \href{https://github.com/KobusMarais/cos301_Longsword_Navigation}{GitHub}
\url{https://github.com/KobusMarais/cos301_Longsword_Navigation}}
\end{center}
\end{titlepage}

%\maketitle



\begingroup



\tableofcontents

\addcontentsline{toc}{section}{Table Of Contents}

\endgroup

\newpage
	\section{Service Contracts}
	
	\paragraph{getRoute()}
		\begin{itemize}
			\item \textbf{Test case 1: Acquiring a route that leads from an indoor to an indoor location.} \\
							This test case used \textit{the entrance to the HB lecture halls} as a starting destination and \textit{the centre of the Client Service Centre} as the ending destination.
							This test case passed, but indoor location were slightly off.
							\\ \textbf{9/10}
							
			\item \textbf{Test case 2: Acquiring a route that leads from an indoor to an outdoor location.} \\
							This test case used \textit{the inside of the IT building} as a starting destination and \textit{the outside of HB} as the ending destination.
							This test case passed.
							\\ \textbf{10/10}
							
			\item \textbf{Test case 3: Acquiring a route that leads from an outdoor to an outdoor location.} \\
							This test case used \textit{the outside of HB} as a starting destination and \textit{the outside of the Theology building} as the ending destination.
							This test case passed.
							\\ \textbf{10/10}
							
			\item \textbf{Test case 4: Acquiring a route that leads from an outdoor to an indoor location.} \\
							This test case used \textit{the outside of Centenary} as a starting destination and \textit{the inside of the Merensky library} as the ending destination. 
							This test case passed, but the indoor location was not entirely accurate.
							\\ \textbf{9/10}
		\end{itemize}
		
	\paragraph{dropPin()}
		\begin{itemize}
			\item \textbf{Test case 1: Drop a pin on an indoor location.} \\\
				This test case used \textit{inside the IT building} as the location to drop the pin
				This test case passed/failed, it did/didn't provide an exception when it failed.
			\\ \textbf{/10}
			\item \textbf{Test case 2: Drop a pin on an outdoor location .} \\
				This test case used \textit{Outside Centenary} as the location to drop the pin
				This test case passed/failed, it did/didn't provide an exception when it failed.
			\\ \textbf{/10}
			\item \textbf{Test case 3: Drop a pin on a location that already has a pin on it.} \\
				This test case used \textit{Outside Brooklyn Mall} as the location to drop the pin
				This test case passed/failed, it did/didn't provide an exception when it failed.
			\\ \textbf{/10}
			\item \textbf{Test case 4: Drop a pin on a location that does not exist (the location is not on campus).} \\
				This test case used \textit{Outside Centenary, which was already used} as the location to drop the pin
				This test case passed/failed, it did/didn't provide an exception when it failed.
			\\ \textbf{/10}

		\end{itemize}
	
	\paragraph{removePin()}
		\begin{itemize}
			\item \textbf{Test case 1: Remove a pin on an indoor location.} \\
				This test case used \textit{Inside the IT Building} as the location to remove the pin
				This test case passed/failed, it did/didn't provide an exception when it failed.
			\\ \textbf{/10}
			\item \textbf{Test case 2: Remove a pin on an outdoor location.} \\
				This test case used \textit{Outside Centenary} as the location to remove the pin
				This test case passed/failed, it did/didn't provide an exception when it failed.
			\\ \textbf{/10}
			\item \textbf{Test case 3: Remove a pin on a location that did not initially have a pin on it.} \\
				This test case used \textit{Outside Thuto} as the location to remove the pin
				This test case passed/failed, it did/didn't provide an exception when it failed.
			\\ \textbf{/10}

		\end{itemize}
	
\newpage

\section{Introduction}

	\section{Non-Functional Requirements}
	
	This section discusses and rates 10 non-functional requirements of the NavUP system.
	Below is the analysis of the Navigation module Interface and all its component classes:
	
	\begin{itemize}
		\item \textbf{Documentation:}
			The module as a whole is well documented. The relavent documentation tools were used effectively. [\textbf{Mark: 10/10}]
			
		\item \textbf{Efficiency:} The module is as efficient as it can be with the current tools and resources used. [\textbf{Mark: 8/10}]
		
		\item \textbf{Maintainability:} The project has good module cohesion, as a result it is easily maintainable. [\textbf{Mark: 8/10}]
		
		\item \textbf{Quality:} The module follows a consistent coding standard throughout all its component classes. The code itself is readable with little effort. [\textbf{Mark: 7/10}]
		
		\item \textbf{Reuseablity:} Due to the explicit requirements of NavUP, not all of the modules are easily reuseable. However slight tweaks to some modules is enough to be able to reuse them effectively. [\textbf{Mark: 5/10}]
		
		\item \textbf{Testability:} Testing the code for this module was very easy especially with the different mock tests made available to us.[\textbf{Mark: 9/10}]
	
		\item \textbf{Complaince:} This project does comply with the specification given to some extent, however there was some missing functionality. [\textbf{Mark: 5/10}]
	
		\item \textbf{Robustness:} The code here does handle a fair amount of exceptions and unexpected termination and thus is quite robust.[\textbf{Mark: 7/10}]

		\item \textbf{Cohesion:} The code is quite coesive as it leads into itself quite well. Modules interact with each other and the code ends up working well with all its parts.[\textbf{Mark: 10/10}]\\

		
	\end{itemize}
	
	
\section{Evaluating Test Cases for Non-functional Requirements}
We will be evaluating the non-functional requirement test cases used by Longsword Navigation
The criteria used is as follows:
\begin{itemize}  
\item The variety of test cases used with regards to a minimum of 3 of the non-functional requirements used during our own testing
\item The usefulness of test cases
\item How efficiently the test cases were used 
\end{itemize}
It was discovered that no apparent non-functional test cases were used by the previous team. It is for this reason that we give the group \textbf{0} for this section \\\\
\textbf{Score : 0/10}


\end{document}
